\documentclass{article}
\usepackage{enumitem}
\usepackage{multicol}
\documentclass{article}
\usepackage{tabularx}
\usepackage{booktabs}
\usepackage[columnsep=1pt]{multicol}
\usepackage[left=0.2in, right=0.2in, top=1in, bottom=1in]{geometry} % Custom margins
\title{A Minimal US Constiution}
\author{@absolutemadlado}

 

\begin{document}
\maketitle

\begin{table}[ht]
\centering
\caption{Comparison of U.S. Constitution vs. Minimal Constitution}
\label{tab:constitution-comp}
\begin{tabularx}{\textwidth}{lccc}
\toprule
\textbf{Feature} & \textbf{U.S. Constitution} & \textbf{Minimal Constitution} & \textbf{Difference} \\
\midrule
Full text & \multicolumn{1}{p{3cm}}{~7,600 words (with 27 amendments)} & \multicolumn{1}{p{3cm}}{~1,874 words} & 75\% shorter (word count) \\
Provisions/Rules & 151 provisions & 86 rules & 43\% fewer rules \\
Flesch-Kincaid Grade & 17.9 (post-college) & 10.2 (10th grade) & 7.7-grade reduction \\
Flesch Reading Ease & 25.4 (very difficult) & 65.1 (plain English) & 39.7-point increase \\
Approximate Age & 22+ years & 15-16+ years & Simplified for youth \\
Sentence Length & Avg. 34 words & Avg. 12 words & 65\% shorter \\
Passive Voice & ~20\% of sentences & ~5\% of sentences & 15\% reduction \\
Vocabulary & Archaic/Latin terms & Modern plain English & Modernized language \\
Comprehension & Requires legal training & High school accessible & Public-friendly design \\
\bottomrule
\end{tabularx}

\smallskip
\footnotesize
\textit{Note:} Readability metrics calculated using standard Flesch-Kincaid analysis. Data reflects constitutional texts as of 2023.
\end{table}
\newpage
---\begin{quote}
\itshape % Applies italics to entire quote
  ``We the People of the United States establish this Constitution to:
  \begin{itemize}[label=-, nosep, leftmargin=5em] % Hyphen list with indentation
	\item Create a fair and effective government,
	\item Protect fundamental freedoms for all,
	\item Ensure peace and justice through collaboration,
	\item Promote the well-being of every generation,
	\item Secure liberty and equality as enduring ideals.''
  \end{itemize}
\end{quote}
\begin{multicols}{2}
\section*{Article I : Branches of Government}

\begin{enumerate}[label=\Alph*)]
    \item \textbf{Legislative Branch.}
    
    Congress.
    
    \begin{enumerate}[label=1.1.\arabic*]
        \item The House of Representatives: Members elected every 2 years by voters in their state. To serve, you must be at least 25 years old, a U.S. citizen for 7 years, and live in the state you represent.
        
        \item The Senate: Two Senators per state, elected by voters for 6-year terms. To serve, you must be at least 30 years old, a U.S. citizen for 9 years, and live in your state.
        
        \item[] \textbf{Powers.}
        
        \item Tax, fund defense, regulate trade, manage immigration, and establish post offices, roads, and standards for weights and measures.
        
        \item Grant patents, copyrights, and regulate bankruptcy.
        
        \item Define and punish crimes at sea, counterfeiting, or against international law.
        
        \item Declare war, approve treaties, and oversee the President’s appointments (e.g., judges, ambassadors).
        
        \item Pass laws necessary to execute these powers.
        
        \item[] \textbf{Creation of laws.}
        
        \item A bill must pass both the House and Senate. If the President rejects it (veto), Congress can override with a 2/3 vote in both parts.
        
        \item No law may Punish actions that were legal when committed (\textit{ex post facto}).
        
        \item No law may Target individuals for punishment without trial (\textit{bill of attainder}).
        
        \item[] \textbf{Impeachment.}
        
        \item The House can accuse (impeach) any federal official of serious crimes or abuses of power. The Senate then holds a trial, with the Chief Justice presiding if the President is impeached. A 2/3 Senate vote removes the official.
        
        \item Officials removed by impeachment may not hold federal office again.
    \end{enumerate}
    
    \item \textbf{Executive Branch.}
    
    The President.
    
    \begin{enumerate}[label=1.2.\arabic*]
        \item Leads the government, commands the military, and enforces laws.
        
        \item Elected every 4 years by the Electoral College: Each state gets votes equal to its Senators + Representatives. The candidate with the most votes wins.
        
        \item Must be at least 35 years old, a natural-born U.S. citizen, and a resident for 14 years.
        
        \item No one can be President more than twice, or more than 10 years if they took over mid-term.
        
        \item[] \textbf{Presidential Powers.}
        
        \item Sign/veto laws, appoint judges and ambassadors (with Senate approval), and negotiate treaties.
        
        \item Pardon federal crimes, except in cases of impeachment.
        
        \item[] \textbf{Succession.}
        
        \item If the President dies, quits, or is removed, the Vice President takes over. If both are unavailable, next in line: Speaker of the House, then Senate leaders.
    \end{enumerate}
    
    \item \textbf{Judicial Branch.}
    
    Supreme Court
    
    \begin{enumerate}[label=1.3.\arabic*]
        \item The highest court. Judges serve for life unless they resign or are removed for misconduct.
        
        \item[] Decides cases involving.
        
        \item Disputes between states or foreign governments.
        
        \item Federal laws, treaties, or constitutional rights.
        
        \item Crimes at sea and lawsuits against ambassadors.
        
        \item Treason against the United States consists only of waging war against the nation or aiding its enemies, and requires testimony from two witnesses.
        
        \item[] \textbf{Lower Courts.}
        
        \item Congress may create courts below the Supreme Court.
        
        \item[] \textbf{Judicial Review.}
        
        \item Courts can strike down laws or actions that violate this Constitution.
    \end{enumerate}
 

\section*{Article II: Rights and Liberties}

\begin{enumerate}[label=\Alph*), start=4]
    \item \textbf{Fundamental Freedoms}
    
    \begin{enumerate}[label=2.1.\arabic*]
        \item[] \textbf{Free Expression.}
        \item Everyone has the right to speak, write, practice their beliefs (or none), gather peacefully, and petition the government.
        
        \item The government cannot favor or punish any religion.
        
        \item[] \textbf{Self-Defense.}
        \item People may own weapons for self-defense, sport, or militia service, subject to laws ensuring public safety and preventing harm to others.  
        \textit{Rationale:} Balances individual rights with societal safety, addressing modern debates over gun control.
        
        \item[] \textbf{Privacy: Homes and Bodies.}
        \item No soldier or officer may stay in your home during peacetime without consent. During war, this requires a law passed by Congress.
        
        \item Government searches of your body, home, or belongings require a warrant based on clear evidence.
        
        \item[] \textbf{Digital Privacy.}
        \item Access to private emails, phone records, or location data requires a warrant.  
        \textit{Rationale:} Modernizes 3rd/4th Amendments for digital age and clarifies wartime exceptions.
    \end{enumerate}
    
    \item \textbf{Fair Treatment Under the Law}
    
    \begin{enumerate}[label=2.2.\arabic*]
        \item[] \textbf{Due Process.}
        \item No one may be jailed, fined, or executed without: \\
        A grand jury’s approval for serious crimes. \\
        A speedy, public trial by jury. \\
        In civil lawsuits over \$20, either party may demand a jury.
        
        \item You cannot be tried twice for the same crime (\textit{double jeopardy}) or forced to testify against yourself.
        
        \item[] \textbf{Cruelty Ban.}
        \item Punishments must fit the crime. Torture, extreme penalties, degrading treatment, or prolonged solitary confinement are forbidden.  
        \textit{Rationale:} Expands "cruel and unusual" to address modern prison conditions.
        
        \item[] \textbf{Equality.}
        \item Slavery and forced labor are illegal except as punishment for a crime duly convicted by a court.
        
        \item All people must be treated equally under the law, regardless of race, sex, religion, wealth, disability, gender identity, sexual orientation, or background.
        
        \item[] \textbf{Property Rights.}
        \item Private property may not be taken for public use without fair compensation and a chance to challenge the taking in court.  
        \textit{Rationale:} Codifies the Takings Clause (5th Amendment) and adds judicial oversight.
    \end{enumerate}
    
    \item \textbf{Voting and Representation}
    
    \begin{enumerate}[label=2.3.\arabic*]
        \item[] \textbf{Right to Vote.}
        \item Every citizen 18 or older can vote.
        
        \item No one can be denied voting rights because of race, sex, age (18+), disability, language, or inability to pay taxes.
        
        \item[] \textbf{Fair Elections.}
        \item Congressional districts must: \\
        Be roughly equal in population. \\
        Respect geographic and communal integrity, not partisan advantage. \\
        Ensure racial, ethnic, and political minorities can elect representatives.  
        \textit{Rationale:} Bans partisan/racial gerrymandering (Rucho v. Common Cause, Voting Rights Act).
    \end{enumerate}
    
    \item \textbf{Additional Guarantees and Clarifications}
    
    \begin{enumerate}[label=2.4.\arabic*]
        \item[] \textbf{Unlisted Rights.}
        \item The rights listed here do not deny or dismiss other rights the people already have. For example: \\
        Privacy in personal choices (e.g., marriage, family, medical decisions, abortion, contraception). \\
        Rights to education, travel, bodily autonomy, and fair treatment in private contracts.
        
        \item[] \textbf{Limits on Government Power.}
        \item The government cannot: \\
        Use the Constitution to justify ignoring basic rights (9th Amendment). \\
        Suspend constitutional rights except during invasion or rebellion, and only as narrowly required.
        
        \item[] \textbf{Equal Citizenship.}
        \item Anyone born or naturalized in the U.S. is a full citizen. States cannot deny citizenship rights or equal protection under the law.
        
        \item Non-citizens legally residing in the U.S. have equal protection in criminal and civil proceedings.
        
        \item[] \textbf{Protections Against Exploitation.}
        \item Workers have the right to fair pay, safe conditions, collective bargaining, and strikes, regulated by Congress.
        
        \item[] \textbf{Direct Democracy.}
        \item Citizens can propose and vote on laws or constitutional changes through petitions and ballot initiatives, provided they do not violate federal rights, treaties, or this Constitution.
    \end{enumerate}
 






\newpage
\section*{Article III: Federal-State System}

\begin{enumerate}[label=\Alph*), start=8]
    \item \textbf{Federal Authority}
    
    \begin{enumerate}[label=3.1.\arabic*]
        \item[] \textbf{Supreme Law:}
        \item This Constitution, federal laws, and treaties are the highest law of the land. State governments, courts, and officials must obey them.
        
        \item[] \textbf{Federal Powers:}
        \item Congress holds exclusive authority to: \\
        Declare war, regulate trade between states and nations, manage immigration, and create currency. \\
        Govern U.S. territories (e.g., Puerto Rico, Guam) and admit new states. \\
        Assume and repay all lawful debts of the United States.
        
        \item The federal government shall: \\
        Protect states from invasion. \\
        Defend states against domestic violence (e.g., insurrections, systemic denial of rights).
    \end{enumerate}
    
    \item \textbf{State Responsibilities}
    
    \begin{enumerate}[label=3.2.\arabic*]
        \item[] \textbf{Full Faith and Credit:}
        \item Every state must honor the laws, court rulings, licenses, and contracts of other states.
        
        \item[] \textbf{Equal Treatment:}
        \item States cannot discriminate against visitors from other states in taxes, rights, or access to courts.
        
        \item[] \textbf{Interstate Compacts:}
        \item States may not form binding agreements with other states or foreign nations without congressional approval.
        
        \item[] \textbf{Extradition:}
        \item States must return fugitives accused of crimes to the state where the crime occurred.
    \end{enumerate}
    
    \item \textbf{State Powers}
    
    \begin{enumerate}[label=3.3.\arabic*]
        \item[] \textbf{Reserved Rights:}
        \item Powers not granted to the federal government by this Constitution belong to the states or the people.
        
        \item[] \textbf{Republican Governance:}
        \item States must uphold a republican form of government, including: \\
        Free, fair, and regular elections open to all citizens. \\
        No laws that undermine federal rights, treaties, or constitutional principles.
        
        \item[] \textbf{Essential Services:}
        \item States may not deny access to healthcare, education, emergency aid, or voting rights based on race, religion, gender, disability, residency, gender identity, or sexual orientation.
    \end{enumerate}
    
    \item \textbf{Equal Citizenship}
    
    \begin{enumerate}[label=3.4.\arabic*]
        \item[] \textbf{Birthright Citizenship:}
        \item Anyone born or naturalized in the U.S. is a citizen. States cannot revoke citizenship or pass laws that: \\
        Discriminate based on race, sex, religion, wealth, gender identity, or sexual orientation.
        
        \item[] \textbf{Non-Citizen Protections:}
        \item Non-citizens legally residing in the U.S. have equal protection in criminal and civil proceedings.
        
        \item[] \textbf{Due Process:}
        \item States must ensure fair trials, equal legal access, and protection from arbitrary detention.
    \end{enumerate}
    
    \item \textbf{Territories and New States}
    
    \begin{enumerate}[label=3.5.\arabic*]
        \item[] \textbf{Territorial Governance:}
        \item Congress governs territories, but residents retain all constitutional rights except voting in federal elections unless admitted as a state.
        
        \item Territories may elect non-voting delegates to Congress.
        
        \item[] \textbf{Statehood Process:}
        \item New states may be admitted by Congress on equal footing with existing states.
        
        \item No state may be split, merged, or dissolved without its consent.
    \end{enumerate}
    
    \item \textbf{Debt and Alliances}
    
    \begin{enumerate}[label=3.6.\arabic*]
        \item[] \textbf{Federal Debt:}
        \item The United States assumes responsibility for all lawful debts incurred under this Constitution.
        
        \item[] \textbf{State Restrictions:}
        \item States may not Coin money, grant titles of nobility, or enter treaties/alliances.
        
        \item States may not Tax interstate commerce or federal property.
    \end{enumerate}
\end{enumerate}


\newpage
\section*{Article IV: Amendments}

\begin{enumerate}[label=\Alph*), start=14]
    \item \textbf{Proposal and Ratification}
    
    \begin{enumerate}[label=4.1.\arabic*]
        \item[] \textbf{Proposing Amendments:}
        \item Amendments may be proposed by: \\
        A two-thirds vote in the House and Senate. \\
        A national convention called by Congress if requested by two-thirds of state legislatures. \\
        Citizens may petition for amendments if supported by 10\% of voters in 25 states.
        
        \item Congress must publish proposed amendments publicly for 90 days before voting.
        
        \item[] \textbf{Ratification Process:}
        \item To become law, amendments must be ratified by three-fourths of: \\
        State legislatures. \\
        State conventions elected specifically for ratification.
        
        \item Congress chooses the ratification method but cannot alter it once states begin voting.
        
        \item[] \textbf{Deadline.}
        \item Congress may set a ratification deadline, not exceeding seven years.
        
        \item One six-month extension is allowed if approved by a majority of states.
        
        \item[] \textbf{Integration:}
        \item Ratified amendments must be merged into the relevant existing article(s).
        
        \item If an amendment affects multiple articles, Congress must update all impacted sections within one year.
    \end{enumerate}
    
    \item \textbf{Limits on Amendments}
    
    \begin{enumerate}[label=4.2.\arabic*]
        \item[] \textbf{Unchangeable Principles:}
        \item No amendment may deny or undermine: \\
        Equal rights based on race, sex, religion, disability, gender identity, or sexual orientation. \\
        Freedom of speech, press, assembly, petition, or voting rights for citizens aged 18+. \\
        The guarantee of a republican form of government (free elections, separation of powers, accountability).
        
        \item[] \textbf{Judicial Review:}
        \item Courts may strike down amendments that violate unchangeable principles.
        
        \item[] \textbf{Repealing Amendments:}
        \item A newer amendment may override an older one only if it explicitly repeals the prior text and follows the same proposal/ratification process.
        
        \item[] \textbf{Territorial Participation:}
        \item Residents of U.S. territories may propose amendments through their non-voting congressional delegates.
        
        \item Territories have advisory ratification votes but do not count toward the three-fourths threshold.
    \end{enumerate}
    
    \item \textbf{Emergency Amendments}
    
    \begin{enumerate}[label=4.3.\arabic*]
        \item[] \textbf{Temporary Provisions:}
        \item In times of verified national emergency (declared by 3/4 of Congress and approved by the Supreme Court), Congress may propose temporary amendments with a three-fourths vote.
        
        \item Temporary amendments expire after two years unless ratified permanently under Section 1.
        
        \item[] \textbf{Suspension of Rights:}
        \item No amendment may suspend constitutional rights except during invasions or rebellions, and only as narrowly required.
    \end{enumerate}
    
    \item \textbf{Public Accountability}
    
    \begin{enumerate}[label=4.4.\arabic*]
        \item[] \textbf{Transparency:}
        \item All proposed amendments, debates, and state ratification votes must be published in real-time on a publicly accessible platform.
        
        \item Congress must hold open hearings for citizen-proposed amendments.
        
        \item[] \textbf{Fraud Prevention:}
        \item States must use verifiable, auditable methods for ratification votes.
    \end{enumerate}
    
\end{enumerate}
\end{multicols}

\end{document}